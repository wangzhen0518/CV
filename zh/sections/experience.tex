\section{研究经历}

\newcommand{\expitem}[7]{
\entrybig{\textbf{#1}}{#2}{
% 第一作者,大语言模型,软件测试
}{}
\vspace{-16pt}
\begin{itemize}[leftmargin=12pt]
    \item 成果:#3
    \item 简介:#4
    \begin{enumerate}[leftmargin=14pt]
        \item #5
        \item #6
        \item #7
    \end{enumerate}
\end{itemize}
}

\outerlist{

\expitem{搜索增强大模型检测代码库错误}{2024.03 -- 2024.05}
{深度学习库漏洞检测算法;EvAFuzz 框架;NeurIPS 2024 一作在投}
{针对深度学习库中潜在的安全漏洞,我们用搜索增强大模型发现漏洞的能力,即用搜索算法引导大模型高效地探索程序空间,生成更可能触发漏洞的程序,从而有效地进行安全检测。}
{提出 EvAFuzz 框架,该框架通过进化算法引导大模型迭代地生成更可能触发漏洞的程序。基线在 PyTorch 上检测到 7 个独特的崩溃漏洞,而 EvAFuzz 能够检测到 9 个;}
{提出反馈机制,让大模型修正生成的无效程序,提高生成结果的有效率。基线在 PyTorch 和 TensorFlow 上分别只有 27.69\% 和 12.28\% 的有效率,而 EvAFuzz 达到了 38.80\% 和 34.04\%;}
{在 PyTorch 和 TensorFlow nightly 版本中发现多个新漏洞。}

\expitem{基于模式匹配的通用量子程序优化框架}{2021.07 -- 2022.04}
{QCIR 量子程序优化框架;QVF 模式匹配算法;ICCAD 2022 学生三作,负责代码实现}
{针对量子相干时长有限的挑战,我们提出了基于模式匹配的通用量子电路优化框架 QCIR。QCIR 支持用户自定义成本模型、自定义门类型,并提供通用模式和折叠模式两种模式类型。}
{设计了新颖的模式描述格式,支持用户自定义成本模型和门类型,并区分通用模式和折叠模式,提高了模式匹配的灵活性和适应性;}
{提出了 QVF 模式匹配算法,实现对同构子图的高效匹配;}
{构建了 QCIR 框架,集成了 QVF 模式匹配算法和预定义的一些模式,在 Qiskit 和 ScaffCC 上进行了评估,平均优化率分别达到了 10\% 和 20\%。}

% \entrybig
% 	{\textbf{编译课程实验}}{}
% 	{导师:李诚}{2021年9月 -- 2022年1月}
% \innerlist{
%     \entry{担任队长}
%     \entry{简单c编译器前端的实现
%     \begin{enumerate}
%         \item 词法分析
%         \item 语义分析
%         \item IR生成
%         \item 简单优化
%     \end{enumerate}
%     }
% }

}

\vspace{-15pt}